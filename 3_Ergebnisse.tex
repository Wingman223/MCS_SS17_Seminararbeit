\section{Auswertung des Reviews}
\label{sec:AuswertungReview}

\subsection{Frage 1: Welche Arten von Qualitätsproblemen können während des Softwarelebenszyklus im Management der Requirements Traceability entstehen?}

Aus den 14 ausgewählten Forschungsliteraturen ging hervor, dass alle aktuelle relevanten Qualitätsprobleme sich auf den Kontext von großen Softwaresystemen bezogen. Die Größe und Komplexität kann dort, ohne adequates Management der Requirements Traceability großen Einfluss auf das Projekt bis hin zum gesamten Projekterfolg haben. Als ein weiterer Faktor kann die sich ändernde Projektumgebung genannt werden, die die Notwendigkeit von Verfolgbarkeit weiter in den Vordergrund gerückt hat.

Die Erkenntnisse zeigen, das die Etablierung von Verfolgbarkeit weniger ein Thema ist als ihr direkter Verfall durch Evolution des Softwaresystems. Ein weiterer Punkt steht in der Wiederherstellung der Qualität der Verfolgbarkeit. Beide Qualitätsprobleme erzeugen hohe Aufwände und Kosten und stehen entgegen dem Nutzen und den Motivationen der eigentlichen Einführung von Verfolgbarkeit. In Bezug zu den identifizierten Motivationen stehen die zentralen Aspekte das Änderungsmanagement durch Evolution, die Qualitätssicherung und Wartbarkeit wie auch die Nachweisbarkeit \& Akzeptanz einzelner Systemfunktionen.

\subsection{Frage 1.1: Welche Arten von Qualitätsproblemen in der Requirements Traceability wurden in der vorhandenen Literatur identifiziert?}

Entgegen den Erwartungen ähnelten sich die Problemstellungen in den 14 ausgewählten Literaturen. Daher wurden diese in der Tabelle \ref{tab:auswertung_probleme_allgemein} zu zwei zentralen Problemen im Management der Requirements Traceability zusammengefasst.

\begin{table}[!ht]
\renewcommand{\arraystretch}{1.3}
\caption{Probleme in der Requirements Traceability}
\label{tab:auswertung_probleme_allgemein}
\centering
\begin{tabularx}{\columnwidth}{@{}sm@{}}
\toprule
1. Evolution von Softwaresystemen & \cite{Saputri2016EnsuringApproach,Bavota2014EnhancingInformation,Spanoudakis2004Rule-basedRelations,Javed2014ACode,Tsuchiya2015InteractiveLogs,Omoronyia2011ExploringTraceability,Lago2009AManagement, Mader2009EnablingRelations,Mder2012TowardsMaintenance,Ghabi2015ExploitingCode} \\
2. Wiederherstellung von Verfolgbarkeit in bestehenden Softwaresystemen & \cite{Mder2007CustomizingProcess,Leuser2010TacklingSpecifications,Merten2016DoData} \\
\bottomrule
\end{tabularx} 
\end{table}

Interessanterweise deckten sich die identifizierten Probleme im Management mit den Erkenntnissen von Walia \& Carver \cite{Walia2009AErrors}, die in ihrer Forschungsarbeit anhand eines SLRs herausgefunden hatten, dass die häufigsten Fehler im Requirements Management auf Grund von

\begin{enumerate}
    \item Inadequatem Änderungsmanagement, eingeschlossen der Analyse der Auswirkung durch Ändern von Anforderungen
    \item Inadequater/Schlechter Requirements Traceability
\end{enumerate}

entstehen. \cite{Walia2009AErrors}

Entsprechende Ursachen zu den Probleme wurden in der Literatur implizit oder explizit genannt, siehe Tabelle \ref{tab:auswertung_probleme_ursachen}.

\begin{table}[!ht]
\renewcommand{\arraystretch}{1.3}
\caption{Ursachen für Probleme}
\label{tab:auswertung_probleme_ursachen}
\centering
\begin{tabularx}{\columnwidth}{@{}mss@{}}
\toprule
Fehlen adequater Toolunterstützung & \cite{Saputri2016EnsuringApproach, Mder2007CustomizingProcess, Spanoudakis2004Rule-basedRelations, Ghabi2015ExploitingCode, Leuser2010TacklingSpecifications, Omoronyia2011ExploringTraceability, Tsuchiya2015InteractiveLogs} \\
Aufwand nicht im Verhältnis zum Nutzen & \cite{Javed2014ACode,Lago2009AManagement,Spanoudakis2004Rule-basedRelations, Omoronyia2011ExploringTraceability, Bavota2014EnhancingInformation, Saputri2016EnsuringApproach, Mder2012TowardsMaintenance, Mader2009EnablingRelations} \\
Fehleinschätzung der Notwendigkeit & \cite{Leuser2010TacklingSpecifications, Ghabi2015ExploitingCode}
Einsatz inadequater Tools & \cite{Merten2016DoData} \\
\bottomrule
\end{tabularx}
\end{table}

Als zentrale Ursache für eine schlechte Qualität im Management der Requirements Traceability wird in 7 Literaturen das \enquote{Fehlen adequater Tooluntersützung} genannt. Entsprechend kann angenommen werden das auf Grund dieses Umstands eine vollständige Erfassung so wahrgenommen wird als stände ihr \enquote{Aufwand nicht im Verhältnis zum Nutzen}.

Aus den zentralen Problemen und Ursachen lassen sich erste Aussagen über Qualitätsprobleme im Management der Verfolgbarkeit ableiten

\begin{itemize}
    \item Fehlen adequater Mechanismen um den Verfall von Verfolgbarkeit durch Evolution zu verhindern
    \item Fehlen adequater Mechanismen um die Qualität von Verfolgbarkeit in bestehenden Spezifikationen wiederherzustellen
\end{itemize}

\subsection{Frage 1.2: Welche Qualitätsprobleme können durch die Analyse ihrer Ursache identifiziert werden?}

Alle Probleme wurden explizit oder implizit im Kontext von großen Softwareprojekten genannt. Die Notwendigkeit von Traceability ergibt sich dort auf Grund der Größe oder Komplexität eines Softwareprojekts und ihres Einflusses auf den Projekterfolg \cite{Lago2009AManagement, Mder2007CustomizingProcess, Leuser2010TacklingSpecifications, Tsuchiya2013RecoveringProducts, Ghabi2015ExploitingCode}. Als zusätzlicher Faktor für die Notwendigkeit spricht eine Veränderung der Anforderungen an die Entwicklung von Softwaresystemen, siehe Tabelle \ref{tab:auswertung_probleme_ursachen_needs}.

\begin{table}[!ht]
\renewcommand{\arraystretch}{1.3}
\caption{Veränderungen in den Projektgegebenheiten}
\label{tab:auswertung_probleme_ursachen_needs}
\centering
\begin{tabularx}{\columnwidth}{@{}ms@{}}
\toprule
Fortschritte in den verfügbaren Technologien & \cite{Saputri2016EnsuringApproach} \\
Höhere Nutzererwartungen & \cite{Saputri2016EnsuringApproach} \\
Evolutionäre Entwicklung von Softwaresystemen & \cite{Saputri2016EnsuringApproach} \\
Häufige Veränderungen in den Anforderungen & \cite{Mder2007CustomizingProcess} \\
Notwendigkeit für sicherheitsrelevante Systeme & \cite{Leuser2010TacklingSpecifications} \\
Implizite Notwendigkeit für die Wartbarkeit eines Systems & \cite{Lago2009AManagement} \\
\bottomrule
\end{tabularx} 
\end{table}

Aus der Extraktion von Motivationen für den Einsatz von Verfolgbarkeit ergaben sich ähnliche Gründe die einen Bezug zu den Veränderungen in Tabelle \ref{tab:auswertung_probleme_ursachen_needs} hatten. Tabelle \ref{tab:auswertung_probleme_ursachen_motivation} zeigt das Ergebnis der Extraktion. Da die Definitionen teils implizit oder natürlichsprachlich waren, wurden die Intentionen den Begrifflichkeiten und Definitionen nach dem Standardkwerk von Pohl \cite[Tab. 30-1]{Pohl2008RequirementsTechniken} zugeordnet.

\begin{table}[!ht]
\renewcommand{\arraystretch}{1.3}
\caption{Motivationen für Probleme}
\label{tab:auswertung_probleme_ursachen_motivation}
\centering
\begin{tabularx}{\columnwidth}{@{}ms@{}}
\toprule
Änderungsmanagement & \cite{Spanoudakis2004Rule-basedRelations, Javed2014ACode, Tsuchiya2013RecoveringProducts, Saputri2016EnsuringApproach, Omoronyia2011ExploringTraceability,Bavota2014EnhancingInformation, Mder2007CustomizingProcess, Mader2009EnablingRelationss, Mder2012TowardsMaintenance, Ghabi2015ExploitingCode} \\
Nachweisbarkeit \& Akzeptanz & \cite{Spanoudakis2004Rule-basedRelations, Bavota2014EnhancingInformation, Leuser2010TacklingSpecifications} \\
Qualitätssicherung, Wartung und Pflege & \cite{Spanoudakis2004Rule-basedRelations, Javed2014ACode, Tsuchiya2013RecoveringProducts, Saputri2016EnsuringApproach, Bavota2014EnhancingInformation, Mder2007CustomizingProcess, Mader2012TowardsMaintenance, Lago2009AManagement} \\
\bottomrule
\end{tabularx} 
\end{table}

Anhand der Erkenntnisse zu Veränderungen in Tabelle \ref{tab:auswertung_probleme_ursachen_needs}, Motivationen in Tabelle \ref{tab:auswertung_probleme_ursachen_motivation} sowie bekannten Ursachen aus Tabelle \ref{tab:auswertung_probleme_ursachen} konnte auf die bereits identifizierten Qualitätsprobleme ein Rückgeschlossen hergestellt werden. Weitere Qualitätsprobleme wurden nicht identifiziert.

\subsection{Frage 1.3: Beeinflussen die gefundenen Probleme die Kriterien in \ref{tab:qualitaet_verfolgbarkeit}?}

Alle identifizierten Probleme beeinflussen mindestens eine der Kriterien für Qualität von Verfolgbarkeit

% Vieleicht noch einfügen der Probleme / Erklärung einzeln?

\subsection{Frage 2: Welche Verfahren existieren im Management der Requirements Traceability zur Behandlung von Qualitätsproblemen?}

In den 14 ausgewählten Literaturen wurden insgesamt 23 Verfahren extrahiert, die die identifizierten Qualitätsprobleme in Frage 1.1 addressieren. Es konnte aber festgestellt werden, das alle Verfahren auf einer Kategorie von Verfahren oder Techniken basierten, die bereits erfolgreich in der Domäne eingesetzt worden sind. Die Kategorien entstammen vorhergegangenen SLRs und Erwähnungen in anderen Literaturen \cite{Tsuchiya2015InteractiveLogs, Omoronyia2011ExploringTraceability, Leuser2010TacklingSpecifications, Spanoudakis2004Rule-basedRelations}. Insgesamt wurden 7 zentrale Kategorien von Verfahren bestimmt die Qualitätsprobleme im Management der Requirements Traceability behandeln.

\subsection{Frage 2.1: Greift die Literatur Prozesse oder Methodiken auf, die zu einer Verbesserung der Qualität beitragen?}

Verschiedenste Verfahren und Techniken wurden in der Literatur genannt, mit dem der Aufwand und die Qualität des Mangements von Requirements Traceability verbessert werden kann:

\begin{table}[htbp]
\renewcommand{\arraystretch}{1.3}
\centering
\begin{threeparttable}
\begin{tabularx}{\columnwidth}{@{}XXl@{}}
\toprule
Verfahren & Fokus Traceability & Quelle  \\ \midrule
1. LISA Ansatz & Architektur zu Code & \cite{Javed2014ACode} \\
2. Unified Process & Architektur zu Design & \cite{Javed2014ACode} \\
3. Rekonstruktion Architektur anhand Code & Code zu Design & \cite{Javed2014ACode} \\
4. ArchTrace & Architektur zu Code & \cite{Javed2014ACode} \\
5. XML-basierter Ansatz für Support traces über UML-basiertes Designmodelle & Design zu Code & \cite{Javed2014ACode} \\
6. MolhadoArch & Architektur zu Code & \cite{Javed2014ACode} \\
7. Reverse Engineering anhand Designspezifikation & Architektur zu Code & \cite{Javed2014ACode} \\
8. Traceability über Archpoints \& Archmappings & Architektur zu Code & \cite{Javed2014ACode} \\
9. Traceability über Views & Design zu Code & \cite{Javed2014ACode} \\
10. Traceability über Architekturspezifikation & Architektur zu Code & \cite{Javed2014ACode} \\
11. Tactic-centric approach & Architektur zu Code & \cite{Javed2014ACode} \\
12. Einsatz IR zur Rekonstruierung von Traces & Anforderungen und Testspezifikation, Evolution & \cite{Leuser2010TacklingSpecifications, Merten2016DoData} \\
13. Ontologiebasierter Ansatz mit IR & Szenarien und Systementwicklung & \cite{Saputri2016EnsuringApproach} \\
14. Rekonstruierung Links mit IR & Anforderungen und Code & \cite{Tsuchiya2013RecoveringProducts} \\
15. traceMaintainer & Design zu Code & \cite{Mader2009EnablingRelations} \\
16. Ergebnisse Studie zu Software Process Level (SPL) & Modelle / Sichten & \cite{Lago2009AManagement} \\
17. Ergebnisse Studie zu Software Process Management & Requirements zu Code & \cite{Lago2009AManagement} \\
18. Einsatz eventbasierter Traceability über Publish/Subscribe Beziehungen & Requirement zu Artefakten & \cite{Mder2012TowardsMaintenance} \\
19. Einsatz regelbasierte Techniken & Anforderung zu Modell & \cite{Mder2012TowardsMaintenance, Spanoudakis2004Rule-basedRelations} \\
20. Traceability anhand unklarheiten über Artefakten & Artefakt zu Code & \cite{Ghabi2015ExploitingCode} \\
21. Verwenden von Bayesian Believe Networks für Traceability & Use Cases zu Code & \cite{Omoronyia2011ExploringTraceability} \\
22. Traceability anhand bestimmung von Relevant eines Links & Use Cases zu Code & \cite{Omoronyia2011ExploringTraceability} \\
23. Bewertung von Ähnlichkeiten zwischen Artefakten mit IR & Artefakte & \cite{Bavota2014EnhancingInformation} \\
\bottomrule
\end{tabularx}
\medskip
    IR : Information Retrieval
    Design : Modellierung
\end{threeparttable}
\caption{Verfahren und Fokus}
\label{tab:auswertung_verfahren_kategorisierung}
\end{table}

Sie alle fußten aber auf ein oder mehreren Verfahren oder Techniken die sich im Einsatzgebiet bereits bewährt haben. In einem vorrangegangen SLR zu Requirements Traceability Ansätzen \cite{Javed2014ACode, Mader2012TowardsMaintenance} wurden bereits schon verfügbare Ansätze kategorisiert. In der Analyse ließen sich ein Großteil der genannten Verfahren den Kategorisierungen zuordnen. Abweichend zu den in den SLRs genannten Kategorisierungen, konnten die in \cite{Tsuchiya2015InteractiveLogs, Omoronyia2011ExploringTraceability, Leuser2010TacklingSpecifications, Spanoudakis2004Rule-basedRelations} beschriebenen Ähnlichkeitsbasierten Verfahren nicht direkt zugeordnet werden. Tabelle \ref{tab:auswertung_verfahren_kategorisierung} zeigt alle Kategorisierungen und ihre Nennungen in den im Review gefundenen Quellen.

\begin{table}[htbp]
\renewcommand{\arraystretch}{1.3}
\caption{Zuordnung Kategorisierung der Verfahren und ihrer Nennungen}
\label{tab:auswertung_verfahren_kategorisierung}
\centering
\begin{threeparttable}
\begin{tabularx}{\columnwidth}{@{}Xl@{}}
\toprule
Grundtechnik & Quelle  \\ \midrule
Linkbasierte Traceability & \cite{Javed2014ACode, Spanoudakis2004Rule-basedRelations, Omoronyia2011ExploringTraceability} \\
Regelbasierte Traceability & \cite{Javed2014ACode, Ghabi2015ExploitingCode, Lago2009AManagement, Mader2012TowardsMaintenance, Mder2007CustomizingProcess, Spanoudakis2004Rule-basedRelations}  \\
Ähnlichkeitsbasierte Traceability & \cite{Tsuchiya2015InteractiveLogs, Omoronyia2011ExploringTraceability, Leuser2010TacklingSpecifications, Spanoudakis2004Rule-basedRelations} \\
Evolutionsbasierte Traceability & \cite{Javed2014ACode, Mader2012TowardsMaintenance} \\
Modellgetriebene Traceability & \cite{Javed2014ACode,Lago2009AManagement, Mader2012TowardsMaintenance, Mder2007CustomizingProcess, Mader2009EnablingRelations, Spanoudakis2004Rule-basedRelations} \\
Information Retrieval (IR) basierte Traceability & \cite{Javed2014ACode, Bavota2014EnhancingInformation, Saputri2016EnsuringApproach, Leuser2010TacklingSpecifications, Merten2016DoData}\\
Machine Learning basierte Traceability & \cite{Javed2014ACode} \\
\bottomrule
\end{tabularx}
\medskip
      %\footnotesize\textbf{Legende:}\smallskip
      %\begin{tablenotes}\footnotesize
      %\item[*] In \cite{Kollanus2010Test-DrivenApproach} werden die gleichen Studien wie in \cite{Kollanus2011CriticalDevelopment} untersucht.
      %\end{tablenotes}
\end{threeparttable}
\end{table}

Für eine genaue Beschreibung der Verfahren und Techniken wird auf \cite{Javed2014ACode, Leuser2010TacklingSpecifications} verwiesen. 

\subsection{Frage 2.2: Addressieren die gefundenen Verfahren die Kriterien in \ref{tab:qualitaet_verfolgbarkeit}?}

Alle identifizierten Probleme addressieren mindestens eine der Kriterien für Qualität von Verfolgbarkeit

\subsection{Frage 3: Lassen sich die Erkenntnisse aus 1-2 zu Aussagen über mögliche Verfahrensweisen im spezifischen Anwendungsfall zusammenfassen?}

Aus den Problemstellungen in Frage 1.1 lassen sich zwei spezifische Anwendungsfälle ableiten

\begin{itemize}
    \item Verhindern des Verfalls von Verfolgbarkeit durch Evolution
    \item Verbesserung der Qualität von Verfolgbarkeit in bestehenden Spezifikationen
\end{itemize}

\textit{1. Verhindern des Verfalls von Verfolgbarkeit durch Evolution}

Bezeichnet alle Maßnahmen die angewendet werden müssen, um die Qualität wiederherzustellen. Eine Wiederherstellung kann nach den Kategorien in Tabelle \ref{tab:auswertung_verfahren_kategorisierung} in zwei Verfahrensarten eingestuft werden. Die explizite Traceability benötigt im Vorfeld der Entwicklungen eine Spezifikation/Modell anhand derer sie Veränderungen im Softwaresystem bei ihrer Entstehung erkennt und propagiert. Die implizite Traceability wird erst zu einem späteren Zeitpunkt ausgeführt und rekonstruiert Beziehungen anhand der ihr gegebenen Informationen. Unter dem Fokus der Qualität ist abzuwägen welche Art von Verfahren verwendet wird.

Handelt es sich um ein neues Softwareprojekt ist die explizite Traceability vorzuziehen, da ihr initialer Aufwand zwar größer ist als der impliziter Verfahren, die Qualität der Verfolgbarkeit aber dauerhaft sichert. Explizite Verfahren sind die \enquote{Modellgetriebene Traceability}, \enquote{Regelbasierte Traceability}, \enquote{Evolutionsbasierte Traceability}.

Handelt es sich um ein bestehendes Softwareprojekt ist eine explizite Traceability nicht unbedingt anwendbar. Da die Rekonstruktion von Informationen ein zeit- und kostenintensiver Prozess ist, ist der Einsatz solcher Tools ratsam. Je nach Projekt wurde bereits schon Verfolgbarkeit bis zu einem bestimmten Grad etabliert. Um die Qualität zu erhöhen werden die verfügbaren Informationen dazu verwendet, weitere Beziehungen herzustellen. Implizite Verfahren sind die \enquote{Ähnlichkeitsbasierte Traceability}, \enquote{Information Retrieval (IR) basierte Traceability}, \enquote{Linkbasierte Traceability}, \enquote{Machine Learning basierte Traceability}

\textit{2. Verbesserung der Qualität von Verfolgbarkeit in bestehenden Spezifikationen}

Entsprechend der Empfehlungen aus 1 ist ein implizites Verfahren zu verwenden, dass anhand verfügbarer Informationen und Beziehungen die Qualität bis zu einem gewissen Grad wiederherstellt.