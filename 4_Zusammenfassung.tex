\section{Diskussion \& Fazit}
\label{sec:DiskussionFazit}

Das Ziel dieser Studie war es, bestehende Qualitätsprobleme im Management der Requirements Traceability zu identifizieren, um diese zukünftig anhand bekannter Verfahren bestimmen und vermeiden zu können.

In diesem Systematic Literature Review wurden 2 zentrale Qualitätsprobleme in der aktuellen Forschung zu Requirements Traceability identifiziert. Zum einen Veränderungen durch die Evolution eines Softwaresystems, zum Anderen die Wiederherstellung von Requirements Traceability in bestehenden Softwaresystemen. Beide Probleme entstanden im Kontext von großen Softwareprojekten auf Grund von Größe, Komplexität sowie veränderten Bedingungen in der Softwareentwicklung. Die Ursache für die Probleme waren auf hohe Zeit- und Kostenaufwände zurückzuführen, die dazu führten, dass die Qualität im Management der Requirements Traceability nur noch schwer zu sichern war. Aus der Forschungsliteratur gingen aber auch 21 Lösungsansätze hervor die diese Problematik addressierten. Auf Grund der Ähnlichkeit und Ergebnissen aus vorrangegangenen SLRs konnten alle Lösungsansätze anhand ihres Grundverfahren oder Technik kategorisiert werden. Insgesamt 7 Kategorien von Verfahren konnten bestimmt werden die dann zu einer Aussage über die Behebbarkeit von Qualitätsproblemen führten. Es wurde zwischen Verfahren unterschieden die implizit oder explizit Qualitätsprobleme im Management behoben, wobei die jeweilige Situation mit betrachtet werden musste. Es kann festgehalten werden, dass die bekannten Verfahren bereits einen Großteil der Traceability über die Evolution und Wiederherstellung abdecken. Doch unterliegen sie für den Anwendungsfall spezifischen Limitierungen die je nach Verfahren abgewägt werden müssen. Wenn anwendbar sind hierbei explizite Verfahren vorzuziehen die auf Modellen oder Spezifikationen basieren, da sie den besten Support über den gesamten Softwarelebenszyklus bieten um den Verfall effektiv zu verhindern. 

Eine Erkenntnis ist das alle genannten Verfahren ein Management von Anforderungen, Design oder Architekturen zur Implementierung bieten und damit die RS und post-RS Phase unterstützen. Es ging aus dem SLR nicht hervor ob die Verfahren auch die pre-RS Phase unterstützen.


