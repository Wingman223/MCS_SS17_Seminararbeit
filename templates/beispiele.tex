% Palatino/Palladio Times Roman

\section{Examples Latex}
% no \IEEEPARstart
This demo file is intended to serve as a ``starter file''
for IEEE conference papers produced under \LaTeX\ using
IEEEtran.cls version 1.8b and later.
% You must have at least 2 lines in the paragraph with the drop letter
% (should never be an issue)
I wish you the best of success.

Kurze Einführung in Latex: \\ \url{https://de.wikibooks.org/wiki/LaTeX-Kompendium:_Schnellkurs:_Erste_Schritte} \\
\url{https://de.wikibooks.org/wiki/LaTeX-Kompendium:_Schnellkurs:_Formatierung}

\hfill mds

\hfill August 26, 2015

\subsection{Subsection Heading Here - Listen}
Subsection text here.

\subsubsection{Subsubsection Heading Here}
Subsubsection text here.

\subsection{Listen}
geordnete Liste
\begin{enumerate}
\item erstes Element
\item zweites Element
\item dritten Element
\end{enumerate}

ungeordnete Liste
\begin{itemize}
\item Punkt A
\item Punkt B
\end{itemize}

verschachtelte Listen
\begin{enumerate}
\item erstes Element
  \begin{itemize}
  \item Unterpunkt A
  \item Unterpunkt B
  \end{itemize}
\item zweites Element
\end{enumerate}

itemize Liste
\begin{itemize}[\IEEEsetlabelwidth{AF1.1}] % “AF1.1” is the longest label in the list
\item[AF1] blah
\item[AF1.1] blah
\item[AF2] blah
\end{itemize}

\subsection{Bilder}
% An example of a floating figure using the graphicx package.
% Note that \label must occur AFTER (or within) \caption.
% For figures, \caption should occur after the \includegraphics.
% Note that IEEEtran v1.7 and later has special internal code that
% is designed to preserve the operation of \label within \caption
% even when the captionsoff option is in effect. However, because
% of issues like this, it may be the safest practice to put all your
% \label just after \caption rather than within \caption{}.
%
% Reminder: the "draftcls" or "draftclsnofoot", not "draft", class
% option should be used if it is desired that the figures are to be
% displayed while in draft mode.
%

Abbildung \ref{fig_Beispiel} wurde automatisch angeordnet.
\begin{figure}[!ht]
\centering
\includegraphics[width=2.5in]{99_Example}
% where an .eps filename suffix will be assumed under latex,
% and a .pdf suffix will be assumed for pdflatex; or what has been declared
% via \DeclareGraphicsExtensions.
\caption{Ein Beispielbild.}
\label{fig_Beispiel}
\end{figure}

% Note that the IEEE typically puts floats only at the top, even when this
% results in a large percentage of a column being occupied by floats.


% An example of a double column floating figure using two subfigures.
% (The subfig.sty package must be loaded for this to work.)
% The subfigure \label commands are set within each subfloat command,
% and the \label for the overall figure must come after \caption.
% \hfil is used as a separator to get equal spacing.
% Watch out that the combined width of all the subfigures on a
% line do not exceed the text width or a line break will occur.
%
%\begin{figure*}[!t]
%\centering
%\subfloat[Case I]{\includegraphics[width=2.5in]{box}%
%\label{fig_first_case}}
%\hfil
%\subfloat[Case II]{\includegraphics[width=2.5in]{box}%
%\label{fig_second_case}}
%\caption{Simulation results for the network.}
%\label{fig_sim}
%\end{figure*}
%
% Note that often IEEE papers with subfigures do not employ subfigure
% captions (using the optional argument to \subfloat[]), but instead will
% reference/describe all of them (a), (b), etc., within the main caption.
% Be aware that for subfig.sty to generate the (a), (b), etc., subfigure
% labels, the optional argument to \subfloat must be present. If a
% subcaption is not desired, just leave its contents blank,
% e.g., \subfloat[].

\subsection{Tabelle}
% An example of a floating table. Note that, for IEEE style tables, the
% \caption command should come BEFORE the table and, given that table
% captions serve much like titles, are usually capitalized except for words
% such as a, an, and, as, at, but, by, for, in, nor, of, on, or, the, to
% and up, which are usually not capitalized unless they are the first or
% last word of the caption. Table text will default to \footnotesize as
% the IEEE normally uses this smaller font for tables.
% The \label must come after \caption as always.
%
\begin{table}[!ht]
% increase table row spacing, adjust to taste
\renewcommand{\arraystretch}{1.3}
% if using array.sty, it might be a good idea to tweak the value of
% \extrarowheight as needed to properly center the text within the cells
\caption{An Example of a Table}
\label{table_example}
\centering
% Some packages, such as MDW tools, offer better commands for making tables
% than the plain LaTeX2e tabular which is used here.
\begin{tabular}{|c|c|}
\hline
One & Two\\
\hline
Three & Four\\
\hline
\end{tabular}
\end{table}

Auf \url{http://www.tablesgenerator.com/} kann eine Tabelle, wie Tabelle \ref{table_example}, mit einem graphischen Editor erstellt werden

% Note that the IEEE does not put floats in the very first column
% - or typically anywhere on the first page for that matter. Also,
% in-text middle ("here") positioning is typically not used, but it
% is allowed and encouraged for Computer Society conferences (but
% not Computer Society journals). Most IEEE journals/conferences use
% top floats exclusively.
% Note that, LaTeX2e, unlike IEEE journals/conferences, places
% footnotes above bottom floats. This can be corrected via the
% \fnbelowfloat command of the stfloats package.

\section{Table spanning two columns of the paper in center}

Tabelle die über zwei Spalten gehen, werden mit einem Asterisk/Sternchen (table*) gekennzeichnet.

\begin{table*}
\centering
\begin{tabular}{|c|c|c||c|c|c|}
\hline
A & B & C & D & E & F\\    \hline
A & B & C & D & E & F\\    \hline
A & B & C & D & E & F\\    \hline
A & B & C & D & E & F\\    \hline
A & B & C & D & E & F\\    \hline
A & B & C & D & E & F\\    \hline
A & B & C & D & E & F\\    \hline
A & B & C & D & E & F\\    \hline
\end{tabular}
\end{table*}


\section{Zitate}

Einfaches Zitat \cite{Madeyski2007OnTests}, Zitat mit Seitenangabe \cite[S. 15]{Erdogmus2005OnProgramming} und mehrere Zitate \cite{Holcombe2008SevenImprovement,Fucci2016ATest-Last,Causevic2013EffectsExperiment}.

Notwendige Felder für die verschiedenen Literaturtypen: \url{https://de.wikipedia.org/wiki/BibTeX#Literaturtypen_.28Entry_Types.29} \\
Liste Online-Literaturgeneratoren: \url{https://de.wikibooks.org/wiki/LaTeX-Kompendium:_Zitieren_mit_BibTeX\#Online-Generierung_von_BibTeX-Eintr.C3.A4gen}

\section{Conclusion}
The conclusion goes here.

% conference papers do not normally have an appendix