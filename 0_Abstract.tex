%\begin{abstract}
%\textit{Kontext und Motivation}:
%Die Requirements Traceability ist ein Teil des Requirements Managements im Requirements Engineering. Sie beschreibt die Notwendigkeit der Pflege von Beziehungen zwischen Anforderungen und beliebigen Artefakten von ihrem Usprung bis zur Anforderung sowie allen nachgelagerten Artefakten im Entwicklungsprozess. Diese Möglichkeit der Verfolgbarkeit ist auch unter dem Namen \enquote{Forward} und \enquote{Backward} Traceability bekannt. Eine Forward- und Backward Traceability kann nur dann sichergestellt werden, wenn eine gerichtete Beziehung (Trace) zwischen Artefakten, Anforderungen und zwischen Anforderungen in einer Anforderungspezifikation (RS) existiert. Eine solche Beziehung wird in einschlägiger Literatur als \enquote{pre-RS}, \enquote{RS} und \enquote{post-RS} beschrieben. Die manuelle Pflege solcher Beziehungen ist fehleranfällig. Besonders die Traceability von post-RS Artefakten ist auf Grund der Dynamik eines Entwicklungsprozesses und des Einsatzes von heterogenen Systemen und Formaten problematisch und kann zu Fehlern, Inkonsistenzen bis hin zu Informationsverlust führen.  \textit{Zentrale Fragen / Probleme}: Aus diesem Anlass ist die Automatisierung der Erkennung dieser Probleme ein wichtiges Forschungsgebiet. Doch ist die post-RS Traceability ein weites Feld und nicht alles lässt sich vollständig- oder teilautomatisieren. \textit{Ziele}: In dieser Arbeit soll anhand eines Systematic Literature Review beleuchtet werden, welche Gebiete der post-RS sich automatisieren lassen und zu welchem Grad. Außerdem soll ergründet werden welche Probleme dabei entstehen und noch ungelöst sind. \textit{Wissenschaftlicher Beitrag}: Anhand der Ergebnisse bietet diese Arbeit einen Überblick über die Automatisierbarkeit von post-RS. Zudem zeigt sie anhand der gefundenen Probleme weitere Forschungsfelder auf.
%\end{abstract}

\begin{abstract}
\textit{Kontext und Motivation}:
Das Management von Requirements Traceability wird in großen Softwareprojekten nur spärlich eingesetzt und beschränkt sich zuweilen auf das klassiche Requirements Engineering. Eine Verfolgbarkeit von Anforderungen ist aber nach bekannter Literatur erst dann gewährleistet wenn eine Anforderung von Ursprung bis zur Anforderung sowie allen nachgelagerten Artefakten im Entwicklungsprozess verfolgt werden kann. Dieser Umstand limitiert aber den eigentlichen Grund für das Management von Requirements Traceability, sogenannten Aspekten. 
\textit{Zentrale Fragen / Probleme}:
Ihr Einsatz ist von großem Nutzen wenn nicht sogar entscheidend für den Projekterfolg von großen Softwareprojekten. Daher ist die Qualität im Sinne von Vollständigkeit, Fehlerfreiheit und Konsistenz der Verfolgbarkeitsinformationen entscheidend für die Verwendbarkeit der Aspekte. Daher besteht die Frage welche Probleme es im Management dieser Verfolgbarkeitsinformationen gibt das diese in der Industrie nicht eingesetzt werden. Auch besteht die Frage ob Verfahren existieren diesen Umstand zu mildern wenn nicht sogar zu beheben.
\textit{Ziele}:
Ziel dieser Arbeit ist es, Probleme im Management der Requirements Traceability zu identifizieren um sie in Zukunft vermeiden wenn nicht sogar beheben zu können.
\end{abstract}