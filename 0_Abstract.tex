\begin{abstract}
\textit{Kontext und Motivation}: Requirements Traceability ist ein Teilgebiet des Requirements Engineerings und bezeichnet die Dokumentation von Beziehungen zwischen Anforderungen und beliebigen Artefakten (Traces) die während einer Softwareentwicklung entstehen. Ihre Pflege verbessert das Verständnis und die Wartbarkeit eines Softwareprodukts für alle beteiligten Stakeholder. Eine lückenlose Pflege von Traces wurde schon in der großen Softwarekrise in den 1960er Jahren als Problem erkannt. Ihre Automatisierung ist auf Grund von heterogenen Systemen und Modellen, sowie der Dynamik einer Softwareeentwicklung schwierig. Gotel \& Finkelstein haben 1994 die Thematik weiter analysiert. In ihrem Werk unterteilen Sie Traceabilty in "pre-Requirements Specification" (pre-RS) und post-Requirements Specification" (post-RS) ein und beschreiben, wieso eine allumfassende Lösung zur Automatisierung von Traceability unwahrscheinlich ist. \textit{Zentrale Fragen / Probleme}: Auch heute noch besteht die Problematik. Doch seit der Analyse von Gotel & Finkelstein wurden Fortschritte in der Forschung gemacht und neue Ansätze entwickelt um die Pflege von Traces weitestgehend zu automatisieren.  \textit{Ziele}: Daher soll in dieser Arbeit, anhand eines Systematic Literature Review (SLR), der aktuelle Forschungsstand im Bereich Traceability ermittelt werden. Dabei besteht der Fokus auf dem von Gotel \& Finkelstein eingeführten pre-RS und post-RS Modell. \textit{Wissenschaftlicher Beitrag}: Diese Arbeit soll einen zusammenfassenden Überblick über den aktuellen Forschungsstand im Bereich Traceability geben und mögliche zukünftige Forschungsfelder aufzeigen.

\end{abstract}