\section{Forschungsmethode}

Im ersten Schritt 

Den Richtlinien in \cite{Keele2007GuidelinesEngineering} folgend, wurde ein Systematisches Literatur Review (SLR) nach Kitchenham et al. durchgeführt. Es umfasst die folgenden Schritte

\begin{itemize}
    \item Formulierung eines Review-Protokolls
    \item Durchführen des Reviews
        \begin{itemize}
            \item Identifizierung von Primärstudien
            \item Evaluierung \& Auswahl
            \item Datenextraktion
            \item Datensynthese
        \end{itemize}
    \item Analyse der Ergebnisse
    \item Auswertung der Ergebnisse
\end{itemize}


Methodologien
Enthält sie eine Art von Automatisierung
Was für eine Unterstützung
Wie trägt sie zur Verbesserung bei
Welche Fehler adressiert sie
Aufwand
Nutzen


\subsection{Forschungsmethode}



\subsection{Umfang}
% Quellen? Annahme?
Zur Kategorie der \enquote{nur mit hohem Kosten- und Zeitaufwand} zu erfassenden Informationen gehören all diejenigen Probleme, die manuell oder mit mit erheblichem Mehraufwand gepflegt werden müssen. Das trifft auf ...


Einschränkung post-RS, Minimaler Scope

\subsection{Forschungsfragen}

Fokus : post-RS-traceability

Herausfinden:
- Was muss automatisiert werden
- Gibt es Modelle / Ansätze für die semi- / automatisierung?
- Welche Probleme sind noch ungelöst? Warum nicht / semi- / voll

Forschungsfragen:
- Welche Forschungsbereiche gibt es?
- Was für Probleme gibt es in den verschiedenen Bereichen der post-RS-traceability?
- Was für Ansätze / Modelle gibt es für die Pflege von post-RS-traceability?
- Welche davon sind automatisierbar? semi- und voll automatisierbar

\subsection{Ablauf (Review, Suchprozess)}

Review planen:

Forschungsfragen
Review Protokoll

Review ausführen:

Suchstrategie, Suchprozess, Methodik
Dokumentierung der Suche

Review Protokoll

Bestehende SLRs
Generellese Forschungsfeld
Zitationen von Gotel \& Finkelstein

\subsection{Kriterien (Inkusion-, Exklusion)}

Was sind die Kriterien für die Inkusion- / Exklusion- von Disserationen

\subsection{Analyse}

Wie wird analysiert, was sind die Methoden?


