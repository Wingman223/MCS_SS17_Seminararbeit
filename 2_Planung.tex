\section{Planung des systematischen Literatur Reviews}
\subsection{Umfang}

Einschränkung post-RS, Minimaler Scope

\subsection{Forschungsfragen}

Fokus : post-RS-traceability

Herausfinden:
- Was muss automatisiert werden
- Gibt es Modelle / Ansätze für die semi- / automatisierung?
- Welche Probleme sind noch ungelöst? Warum nicht / semi- / voll

Forschungsfragen:
- Welche Forschungsbereiche gibt es?
- Was für Probleme gibt es in den verschiedenen Bereichen der post-RS-traceability?
- Was für Ansätze / Modelle gibt es für die Pflege von post-RS-traceability?
- Welche davon sind automatisierbar? semi- und voll automatisierbar

\subsection{Ablauf (Review, Suchprozess)}

Review planen:

Forschungsfragen
Review Protokoll

Review ausführen:

Suchstrategie, Suchprozess, Methodik
Dokumentierung der Suche

Review Protokoll

Bestehende SLRs
Generellese Forschungsfeld
Zitationen von Gotel \& Finkelstein

\subsection{Kriterien (Inkusion-, Exklusion)}

Was sind die Kriterien für die Inkusion- / Exklusion- von Disserationen

\subsection{Analyse}

Wie wird analysiert, was sind die Methoden?


