Als Requirements Traceability bezeichnet man im Requirements Engineering die Nachvollziehbarkeit von Anforderungen über den gesamten Sofware Lebenszyklus.

Diese lässt sich in zwei Phasen, die pre-Requirements Spezifikation (pre-RS) Traceability und post-Requirements Spezifikation (post-RS) Traceability unterteilen. 

In der Praxis gestaltet sich die Pflege von Traces über den gesamtent Software Lebenszyklus als schwierig. Die zunehmende heterogenität 

In der Praxis gestaltet sich die Pflege von Traces, auf Grund der zunehmenden heterogenität von Systemen, als schwierig.

---

Die Requirements Traceability beschreibt im Requirements Engineering die Nachvollziehbarkeit von Anforderungen über den gesamten Software Lebenszyklus. Dieser lässt sich in zwei Phasen, die pre-Requirements Spezifikation (pre-RS) Traceability und post-Requirements Spezifikation (post-RS) aufteilen.

---

Das Ziel von Software Traceability ist die Dokumentation von relevanten Beziehungen zwischen Artefakten, die während eines Software Lebenszyklus entstanden sind. Ihre Pflege ermöglichen die Nachvollziehbarkeit von Anforderungen bis hin zur Implementierung und verbessern so das Verständnis und die Wartbarkeit eines Softwareprodukts für alle beteiligten Stakeholder. In der Praxis macht die zunehmende heterogenität von Systemen eine durchgehende Pflege schwierig. Nebst Pflege von Anforderungen in verschiedenen Repräsentationsformen wird zunehmend der Einsatz von unstrukturierten Issue Tracker Systemen (ITS) populär. Die dort gepflegte Information ist schlecht strukturiert und enthält lange Diskussionen, auch 'noise' genannt, mit teils relevantem Informationsgehalt zu einem Ticket. Daher ist der Einsatz von Algorithmen nur begrenz möglich da diese klar strukturierte Informationen benötigten. Um entstandenen 'noise' aus den Daten eines ITS zu filtern haben sich Information Retrieval (IR) im Praxiseinsatz bewährt. Es existieren aber auch Machine Learning Ansätze um die Qualität und das Wiederauffinden von Trace Links zu verbessern.

\textit{Kontext und Motivation}:
Requirements Traceability ist ein Teilgebiet des Requirements Engineerings und bezeichnet die Dokumentation von Beziehungen zwischen Anforderungen und beliebigen Artefakten (Traces) die während einer Softwareentwicklung entstehen. Ihr Pflege verbessert das Verständnis und die Wartbarkeit eines Softwareprodukts für alle beteiligten Stakeholder.

Die manuelle Dokumentation ist 


Eine lückenlose Pflege gestaltet sich in der Praxis aber auf Grund der zunehmenden heterogenität von Systemen sowie der Dynamik eines Softwareprojekts als 

Eine lückenlose Dokumentation gestaltet sich in der Praxis aber auf Grund der zunehmenden heterogenität von Systemen sowie der Dynamik eines Softwareprojekts als schwierig. Daher ist eine Automatisierung von Traceability unerlässlich um Fehler und Inkonsistenzen zu vermeiden. Die gegebene Problematik haben Gotel & Finkelstein 1994 bereits Ana


Daher wurde bereits schon einige Forschungsarbeit im Feld geleistet um die Qualität von Traces zu verbessern.





Requirements Traceability, Requirements Engineering, Dokumentation von Beziehungen zwischen Artefakten, Verbesserung Verständnis und Wartbarkeit, Einsatz im Projekt, Lückenlose Pflege manuell schwierig, heterogene Systeme & Dynamik, Automatisierung unerlässlich, Inkonsistenzen und Fehler, Forschungsarbeit, Gotel & Finkelstein, pre-RS post-RS, Forward, Backward, Between RS <->,  Horizontal, Vertikal, Rationale?

\textit{Zentrale Fragen / Probleme}:
Forschungsfelder, Automatisierung, Horizontal, Vertikal, pre-RS, post-RS, Forschungsstand,

\textit{Ziele}
SLR, Fokussierung, Vertikal, Horizontal, Forschungsstand, 

\textit{Wissenschaftlicher Beitrag}
Überblick Forschungsstand, zukünftige Forschungsfelder
















Requirements Traceability ist ein Teilgebiet des Requirements Engineerings und bezeichnet die Dokumentation von Beziehungen zwischen Anforderungen und beliebigen Artefakten (Traces) die während einer Softwareentwicklung entstehen. Ihr Pflege verbessert das Verständnis und die Wartbarkeit eines Softwareprodukts für alle beteiligten Stakeholder.

Eine lückenlose Dokumentation von Traces ist auf Grund von heterogenen Systemen sowie der Dynamik eines Softwareprojekts schwierig. Daher ist eine Automatisierung unerlässlich um Fehler und Inkonsistenzen frühzeitig zu erkennen. 

Daher wurde bereits im Feld Forschungsarbeit geleistet um die Qualität von Traces zu verbessern.

\textit{Zentrale Fragen / Probleme}:


Tro

Auf Grund von heterogenen Systemen sowie der Dynamik eines Softwareprojekts, ist die manuelle Pflege von Traces problematisch und fehleranfällig. Daher wurde bereits im Feld Forschungsarbeit geleistet um 



Eine manuelle Dokumentation is

Eine manuelle Dokumentation von Traces ist auf Grund der Dynamik eines Softwareprojekts schwierung und kann zu Informationsver

Ihre manuelle Pflege jedoch, ist auf Grund der Dynamik eines Softwareprojekts schwierig und kann zu Informationsverlust führen. Daher wurde bereits Forschungsarbeit im Feld geleistet um die gegebenen Probleme näher zu beleuchten und mögliche Lösungen zu erarbeiten.

So hatten Gotel & Finkelstein eine automatisierbarkeit 

die  hätte lohnend im Projekt genutzt werden können.

Auf Grund von heterogenen Systemen sowie der Dynamik eines Softwareprojekts, ist die manuelle Pflege von Traces problematisch und fehleranfällig. Daher wurde bereits im Feld Forschungsarbeit geleistet um 

Die Pflege von Traces über den gesamten Softwarelebenszyklus erweist sich auf Grund von heterogenen Systemen und der Dynamik einer Softwareentwicklung als schwierig, was im Gegenzug dazu führen kann, dass entsprechende Informationen nicht mehr lohnend im Projekt genutzt werden können. Daher ist eine Automatisierung von Traceability ein wichtiger Faktor um Fehler und Inkonsistenzen frühzeitig zu erkennen.

\textit{Zentrale Fragen / Probleme}:
Die gegebene Problematik wurde in der einschlägigen Literatur bereits näher beleuchtet. So lässt sich Traceablity in horizontale und vertikale Traceability einteilen. 

So kann Traceability in pre-Requirements (pre-RS), post-Requirements (post-RS) und 

Die Beschriebene Problemaitk


mit dem Ziel, das Verständnis und die Wartbarkeit eines Softwareprodukts für alle beteiligten Stakeholder zu verbessern.

Die Pflege von Traces über den gesamten Softwarelebenszyklus erweist sich auf Grund von heterogenen Systemen und der Dynamik einer Softwareentwicklung als schwierig.  Daher ist die automatisierte Erstellung und Pflege von Traces unerlässlich um Fehler und Inkonsistenten frühstmöglich zu erkennen,

---------------------------


Kontext
Requirements Traceability, Verfolgbarkeit von Anforderungen, Disizplin Requirements Engineering, Anforderung über gesamten Lebenszyklus System nachvollziehen, Nutzen - Nachweis Wartung Identifikation Analyse, Klassifikation pre-RS, post-RS, zwischen Anforderungen (Gotel & Finkelstein ), Vorwärts, Rückwärts gerichtet, Zwischen Anforderungen - Typen generalisierung etc, Erweiterte Kontext <-> Anforderungsartefakt <-> Nachgelagerte Artefakte, Rationale? \cite{Pohl2015BasiswissenIREB-Standard,Pohl2008RequirementsTechniken, Pohl2010RequirementsTechniques}

Motivation:
Einsatz: sobald es unmöglich ist alle Anforderungen zu überblicken, Manuelle pflege fehleranfällig, Inkonsistenzen, Gefährlich, , voll- & semiautomatisierung wichtig um Fehler zu vermeiden, 

------------------------------

Die Verfolgbarbarkeit von Anforderungen (auch Nachvollziehbarkeit, Requirements Traceability) ist eine Disziplin des Anforderungsmanagements im Requirements Engineering. Sie beschreibt die Nachvollziehrbarkeit von Anforderungen über den gesamten Lebenszyklus eines Systems. Ihre Pflege unterstützt die Systementwicklung und ebnet den Weg für den Einsatz von Techniken zur Informationsgewinnung.

Eine Anforderung ist Nachvollziehbar wenn ihr Usprung, ihre verschiedenen Verfeinerungs und Spezifkationsschritte als auch ihre nachgelagerten Entwicklungsartefakte verfolgt werden können.

%als auch ihre weitere Verwendung im Entwicklungsprozess nachvollzogen werden kann \cite[S.505]{Pohl2010RequirementsTechniques}. 

Dies lässt sich weiter differenzieren in pre-RS, post-RS und "Traceability zwischen Anforderungen" einteilen \cite{Gotel1994AnProblem}

%In einschlägiger Literatur kann die Verfolgbarkeit von Anforderungen in pre-RS, post-RS und Traceability zwischen Anforderungen eingeteilt werden. Diese Beschreiben die verschiedenen Phasen 

%\textit{Kontext und Motivation}: Requirements Traceability ist ein Teilgebiet des Requirements Engineerings und bezeichnet die Dokumentation von Beziehungen zwischen Anforderungen und beliebigen Artefakten (Traces) die während einer Softwareentwicklung entstehen. Ihre Pflege verbessert das Verständnis und die Wartbarkeit eines Softwareprodukts für alle beteiligten Stakeholder. Eine lückenlose Pflege von Traces wurde schon in der großen Softwarekrise in den 1960er Jahren als Problem erkannt. Ihre Automatisierung ist auf Grund von heterogenen Systemen und Modellen, sowie der Dynamik einer Softwareeentwicklung schwierig. Gotel \& Finkelstein haben 1994 die Thematik weiter analysiert. In ihrem Werk unterteilen Sie Traceabilty in "pre-Requirements Specification" (pre-RS) und post-Requirements Specification" (post-RS) ein und beschreiben, wieso eine allumfassende Lösung zur Automatisierung von Traceability unwahrscheinlich ist.

%Fokus : pre-RS, post-RS, Vertikal, Horizontal <= Forschungsfragen

 % Problem / Fragestellung:


%Verschiedene Modelle, Arten der Repräsentation, manuell, semi- oder vollautomatisch, Gotel & Finkelstein allumfassende Lösung unwarscheinlich,
Felder in denen eine semi- oder vollautomatisierung möglich aber schwierig, Forschungsfragen, herausfinden von Problemen & Forschungsfelder

-------------

\textit{Kontext und Motivation}:

Post-requirements specification (post-RS), beschreibt, im Teilgebiet Requirements Traceability des Requirements Engineering, die Pflege von Beziehungen zwischen Anforderungen und nachgelagerten Artefakten im Softwarelebenszyklus. Ihre Pflege ermöglicht die Verfolgbarkeit von Anforderungen und ihrer Realisierung im Entwicklungsprozess. Nach Gotel und Finkelstein 1994 ist Post RS neben Pre RS einer der zwei Typen, die das Requirement Traceability darstellen.  

verschiedene Artefakte strukturiert, unstrukturiert
Entwicklungsprozess
welche können als Trace automatisiert werden
Wo geht es nur semi automatisiert
Welche Ansätze Modelle, welche Bereiche

Requirements Traceability ist ein Teilgebiet des Requirements Engineerings und bezeichnet die Dokumentation von Beziehungen zwischen Anforderungen und beliebigen Artefakten (Traces) die während einer Softwareentwicklung entstehen. Ihre Pflege verbessert das Verständnis und die Wartbarkeit eines Softwareprodukts für alle beteiligten Stakeholder. Eine lückenlose Pflege von Traces wurde schon in der großen Softwarekrise in den 1960er Jahren als Problem erkannt. Ihre Automatisierung ist auf Grund von heterogenen Systemen und Modellen, sowie der Dynamik einer Softwareeentwicklung schwierig. Gotel \& Finkelstein haben 1994 die Thematik weiter analysiert. In ihrem Werk unterteilen Sie Traceabilty in "pre-Requirements Specification" (pre-RS) und post-Requirements Specification" (post-RS) und beschreiben wieso eine allumfassende Lösung zur Automatisierung von Traceability unwahrscheinlich ist. \textit{Zentrale Fragen / Probleme}: Auch heute noch besteht die Problematik. Doch seit der Analyse von Gotel \& Finkelstein wurden Fortschritte in der Forschung gemacht und neue Ansätze entwickelt um die Pflege von Traces weitestgehend zu automatisieren.  

In dieser Arbeit soll anhand eines Systematic Literature Review (SLR) der aktuelle Forschungsstand im Bereich Traceability ermittelt werden. Hierbei liegt der Fokus

\textit{Ziele}: Daher soll in dieser Arbeit, anhand eines Systematic Literature Review (SLR) der aktuelle Forschungsstand im Bereich Traceability ermittelt werden. Dabei liegt der Fokus auf dem von Gotel \& Finkelstein eingeführten pre-RS und post-RS Modell. \textit{Wissenschaftlicher Beitrag}: Diese Arbeit soll einen zusammenfassenden Überblick über den aktuellen Forschungsstand im Bereich Traceability geben und mögliche zukünftige Forschungsfelder aufzeigen.