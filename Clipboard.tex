Als Requirements Traceability bezeichnet man im Requirements Engineering die Nachvollziehbarkeit von Anforderungen über den gesamten Sofware Lebenszyklus.

Diese lässt sich in zwei Phasen, die pre-Requirements Spezifikation (pre-RS) Traceability und post-Requirements Spezifikation (post-RS) Traceability unterteilen. 

In der Praxis gestaltet sich die Pflege von Traces über den gesamtent Software Lebenszyklus als schwierig. Die zunehmende heterogenität 

In der Praxis gestaltet sich die Pflege von Traces, auf Grund der zunehmenden heterogenität von Systemen, als schwierig.

---

Die Requirements Traceability beschreibt im Requirements Engineering die Nachvollziehbarkeit von Anforderungen über den gesamten Software Lebenszyklus. Dieser lässt sich in zwei Phasen, die pre-Requirements Spezifikation (pre-RS) Traceability und post-Requirements Spezifikation (post-RS) aufteilen.

---

Das Ziel von Software Traceability ist die Dokumentation von relevanten Beziehungen zwischen Artefakten, die während eines Software Lebenszyklus entstanden sind. Ihre Pflege ermöglichen die Nachvollziehbarkeit von Anforderungen bis hin zur Implementierung und verbessern so das Verständnis und die Wartbarkeit eines Softwareprodukts für alle beteiligten Stakeholder. In der Praxis macht die zunehmende heterogenität von Systemen eine durchgehende Pflege schwierig. Nebst Pflege von Anforderungen in verschiedenen Repräsentationsformen wird zunehmend der Einsatz von unstrukturierten Issue Tracker Systemen (ITS) populär. Die dort gepflegte Information ist schlecht strukturiert und enthält lange Diskussionen, auch 'noise' genannt, mit teils relevantem Informationsgehalt zu einem Ticket. Daher ist der Einsatz von Algorithmen nur begrenz möglich da diese klar strukturierte Informationen benötigten. Um entstandenen 'noise' aus den Daten eines ITS zu filtern haben sich Information Retrieval (IR) im Praxiseinsatz bewährt. Es existieren aber auch Machine Learning Ansätze um die Qualität und das Wiederauffinden von Trace Links zu verbessern.